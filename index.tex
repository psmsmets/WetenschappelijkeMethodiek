% Options for packages loaded elsewhere
\PassOptionsToPackage{unicode}{hyperref}
\PassOptionsToPackage{hyphens}{url}
\PassOptionsToPackage{dvipsnames,svgnames,x11names}{xcolor}
%
\documentclass[
  letterpaper,
  DIV=11,
  numbers=noendperiod]{scrreprt}

\usepackage{amsmath,amssymb}
\usepackage{iftex}
\ifPDFTeX
  \usepackage[T1]{fontenc}
  \usepackage[utf8]{inputenc}
  \usepackage{textcomp} % provide euro and other symbols
\else % if luatex or xetex
  \usepackage{unicode-math}
  \defaultfontfeatures{Scale=MatchLowercase}
  \defaultfontfeatures[\rmfamily]{Ligatures=TeX,Scale=1}
\fi
\usepackage{lmodern}
\ifPDFTeX\else  
    % xetex/luatex font selection
\fi
% Use upquote if available, for straight quotes in verbatim environments
\IfFileExists{upquote.sty}{\usepackage{upquote}}{}
\IfFileExists{microtype.sty}{% use microtype if available
  \usepackage[]{microtype}
  \UseMicrotypeSet[protrusion]{basicmath} % disable protrusion for tt fonts
}{}
\makeatletter
\@ifundefined{KOMAClassName}{% if non-KOMA class
  \IfFileExists{parskip.sty}{%
    \usepackage{parskip}
  }{% else
    \setlength{\parindent}{0pt}
    \setlength{\parskip}{6pt plus 2pt minus 1pt}}
}{% if KOMA class
  \KOMAoptions{parskip=half}}
\makeatother
\usepackage{xcolor}
\setlength{\emergencystretch}{3em} % prevent overfull lines
\setcounter{secnumdepth}{5}
% Make \paragraph and \subparagraph free-standing
\ifx\paragraph\undefined\else
  \let\oldparagraph\paragraph
  \renewcommand{\paragraph}[1]{\oldparagraph{#1}\mbox{}}
\fi
\ifx\subparagraph\undefined\else
  \let\oldsubparagraph\subparagraph
  \renewcommand{\subparagraph}[1]{\oldsubparagraph{#1}\mbox{}}
\fi


\providecommand{\tightlist}{%
  \setlength{\itemsep}{0pt}\setlength{\parskip}{0pt}}\usepackage{longtable,booktabs,array}
\usepackage{calc} % for calculating minipage widths
% Correct order of tables after \paragraph or \subparagraph
\usepackage{etoolbox}
\makeatletter
\patchcmd\longtable{\par}{\if@noskipsec\mbox{}\fi\par}{}{}
\makeatother
% Allow footnotes in longtable head/foot
\IfFileExists{footnotehyper.sty}{\usepackage{footnotehyper}}{\usepackage{footnote}}
\makesavenoteenv{longtable}
\usepackage{graphicx}
\makeatletter
\def\maxwidth{\ifdim\Gin@nat@width>\linewidth\linewidth\else\Gin@nat@width\fi}
\def\maxheight{\ifdim\Gin@nat@height>\textheight\textheight\else\Gin@nat@height\fi}
\makeatother
% Scale images if necessary, so that they will not overflow the page
% margins by default, and it is still possible to overwrite the defaults
% using explicit options in \includegraphics[width, height, ...]{}
\setkeys{Gin}{width=\maxwidth,height=\maxheight,keepaspectratio}
% Set default figure placement to htbp
\makeatletter
\def\fps@figure{htbp}
\makeatother
\newlength{\cslhangindent}
\setlength{\cslhangindent}{1.5em}
\newlength{\csllabelwidth}
\setlength{\csllabelwidth}{3em}
\newlength{\cslentryspacingunit} % times entry-spacing
\setlength{\cslentryspacingunit}{\parskip}
\newenvironment{CSLReferences}[2] % #1 hanging-ident, #2 entry spacing
 {% don't indent paragraphs
  \setlength{\parindent}{0pt}
  % turn on hanging indent if param 1 is 1
  \ifodd #1
  \let\oldpar\par
  \def\par{\hangindent=\cslhangindent\oldpar}
  \fi
  % set entry spacing
  \setlength{\parskip}{#2\cslentryspacingunit}
 }%
 {}
\usepackage{calc}
\newcommand{\CSLBlock}[1]{#1\hfill\break}
\newcommand{\CSLLeftMargin}[1]{\parbox[t]{\csllabelwidth}{#1}}
\newcommand{\CSLRightInline}[1]{\parbox[t]{\linewidth - \csllabelwidth}{#1}\break}
\newcommand{\CSLIndent}[1]{\hspace{\cslhangindent}#1}

\KOMAoption{captions}{tableheading}
\makeatletter
\makeatother
\makeatletter
\@ifpackageloaded{bookmark}{}{\usepackage{bookmark}}
\makeatother
\makeatletter
\@ifpackageloaded{caption}{}{\usepackage{caption}}
\AtBeginDocument{%
\ifdefined\contentsname
  \renewcommand*\contentsname{Inhoudsopgave}
\else
  \newcommand\contentsname{Inhoudsopgave}
\fi
\ifdefined\listfigurename
  \renewcommand*\listfigurename{Lijst van figuren}
\else
  \newcommand\listfigurename{Lijst van figuren}
\fi
\ifdefined\listtablename
  \renewcommand*\listtablename{Lijst van tabellen}
\else
  \newcommand\listtablename{Lijst van tabellen}
\fi
\ifdefined\figurename
  \renewcommand*\figurename{Figuur}
\else
  \newcommand\figurename{Figuur}
\fi
\ifdefined\tablename
  \renewcommand*\tablename{Tabel}
\else
  \newcommand\tablename{Tabel}
\fi
}
\@ifpackageloaded{float}{}{\usepackage{float}}
\floatstyle{ruled}
\@ifundefined{c@chapter}{\newfloat{codelisting}{h}{lop}}{\newfloat{codelisting}{h}{lop}[chapter]}
\floatname{codelisting}{Listing}
\newcommand*\listoflistings{\listof{codelisting}{Lijst van listings}}
\makeatother
\makeatletter
\@ifpackageloaded{caption}{}{\usepackage{caption}}
\@ifpackageloaded{subcaption}{}{\usepackage{subcaption}}
\makeatother
\makeatletter
\@ifpackageloaded{tcolorbox}{}{\usepackage[skins,breakable]{tcolorbox}}
\makeatother
\makeatletter
\@ifundefined{shadecolor}{\definecolor{shadecolor}{rgb}{.97, .97, .97}}
\makeatother
\makeatletter
\makeatother
\makeatletter
\makeatother
\ifLuaTeX
\usepackage[bidi=basic]{babel}
\else
\usepackage[bidi=default]{babel}
\fi
\babelprovide[main,import]{dutch}
% get rid of language-specific shorthands (see #6817):
\let\LanguageShortHands\languageshorthands
\def\languageshorthands#1{}
\ifLuaTeX
  \usepackage{selnolig}  % disable illegal ligatures
\fi
\IfFileExists{bookmark.sty}{\usepackage{bookmark}}{\usepackage{hyperref}}
\IfFileExists{xurl.sty}{\usepackage{xurl}}{} % add URL line breaks if available
\urlstyle{same} % disable monospaced font for URLs
\hypersetup{
  pdftitle={Wetenschappelijke Methodiek},
  pdfauthor={Pieter Smets},
  pdflang={nl},
  colorlinks=true,
  linkcolor={blue},
  filecolor={Maroon},
  citecolor={Blue},
  urlcolor={Blue},
  pdfcreator={LaTeX via pandoc}}

\title{Wetenschappelijke Methodiek}
\author{Pieter Smets}
\date{2023-06-12}

\begin{document}
\maketitle
\ifdefined\Shaded\renewenvironment{Shaded}{\begin{tcolorbox}[interior hidden, frame hidden, boxrule=0pt, breakable, enhanced, sharp corners, borderline west={3pt}{0pt}{shadecolor}]}{\end{tcolorbox}}\fi

\renewcommand*\contentsname{Inhoudsopgave}
{
\hypersetup{linkcolor=}
\setcounter{tocdepth}{2}
\tableofcontents
}
\bookmarksetup{startatroot}

\hypertarget{voorwoord}{%
\chapter*{Voorwoord}\label{voorwoord}}
\addcontentsline{toc}{chapter}{Voorwoord}

\markboth{Voorwoord}{Voorwoord}

This is a Quarto book.

To learn more about Quarto books visit
\url{https://quarto.org/docs/books}.

\bookmarksetup{startatroot}

\hypertarget{introductie}{%
\chapter{Introductie}\label{introductie}}

\hypertarget{methode-of-methodiek}{%
\section{Methode of methodiek?}\label{methode-of-methodiek}}

Het verschil tussen methoden en methodieken ligt in het niveau van
abstractie en toepassing.

Methoden verwijzen naar specifieke procedures, technieken of stappen die
worden gebruikt om een bepaald doel te bereiken. Het zijn concrete en
gedetailleerde benaderingen die worden toegepast binnen een bepaald
vakgebied of domein. Een methode kan bijvoorbeeld een statistische
analysemethode zijn, een laboratoriumtechniek of een interviewprotocol.
Methoden zijn praktische instrumenten die worden gebruikt om specifieke
taken uit te voeren.

Aan de andere kant hebben methodieken betrekking op bredere raamwerken
of systemen van methoden die worden toegepast in een bepaalde context.
Een methodologie is een meer overkoepelende en conceptuele benadering
die richtlijnen, principes en strategieën biedt voor het gebruik van
verschillende methoden om een specifiek doel te bereiken. Het omvat vaak
een reeks richtlijnen, procedures en regels die worden toegepast op een
bepaald domein of onderzoeksgebied. Een methodologie kan bijvoorbeeld de
wetenschappelijke methode zijn, een kwalitatieve onderzoeksbenadering
zoals fenomenologie, of een projectmanagementmethodologie zoals Agile of
Waterfall.

Kort gezegd: methoden zijn de specifieke tools en technieken die worden
gebruikt om taken uit te voeren, terwijl methodieken bredere raamwerken
en systemen van methoden zijn die worden toegepast binnen een bepaalde
context om een specifiek doel te bereiken. Methodieken bieden
richtlijnen en strategieën voor het gebruik van methoden in een
samenhangende en gestructureerde manier.

\hypertarget{wetenschappelijke-methode}{%
\section{Wetenschappelijke Methode}\label{wetenschappelijke-methode}}

De wetenschappelijke methode is een systematische benadering die wordt
gebruikt in de wetenschap om kennis te vergaren en hypothesen te testen.
Het omvat een reeks stappen die wetenschappers volgen om betrouwbare en
objectieve resultaten te verkrijgen. Over het algemeen bestaat de
wetenschappelijke methode uit de volgende stappen:

\begin{enumerate}
\def\labelenumi{\arabic{enumi}.}
\tightlist
\item
  Observatie: Waarnemen van een fenomeen of verschijnsel en het stellen
  van een vraag over hoe of waarom dit gebeurt.
\item
  Hypothese: Formuleren van een verklaring of een voorlopige antwoord op
  de vraag die gebaseerd is op bestaande kennis en logisch redeneren.
\item
  Voorspelling: Afleiden van specifieke voorspellingen uit de hypothese
  die getest kunnen worden.
\item
  Experiment: Ontwerpen en uitvoeren van gecontroleerde experimenten om
  de voorspellingen te testen. Het experiment moet zorgvuldig worden
  opgezet, met controlevariabelen en experimentele groepen, om
  nauwkeurige en reproduceerbare resultaten te verkrijgen.
\item
  Data-analyse: Verzamelen en analyseren van de gegevens die tijdens het
  experiment zijn verzameld, vaak met behulp van statistische methoden.
\item
  Conclusie: Op basis van de resultaten van het experiment, beoordelen
  of de hypothesen worden ondersteund of weerlegd. Het kan nodig zijn om
  het experiment te herhalen of verder onderzoek uit te voeren om tot
  een definitieve conclusie te komen.
\item
  Communicatie: Het rapporteren van de resultaten, inclusief de methode,
  bevindingen en conclusies, aan de wetenschappelijke gemeenschap via
  publicaties, presentaties en discussies.
\end{enumerate}

De wetenschappelijke methode is een iteratief proces, waarbij nieuwe
observaties en experimenten kunnen leiden tot herziening van hypothesen
en nieuwe vragen, waardoor de wetenschappelijke kennis voortdurend
groeit en evolueert.

\hypertarget{fair-data-principes}{%
\section{FAIR-data principes}\label{fair-data-principes}}

De FAIR-data principes zijn een set richtlijnen die zijn ontwikkeld om
de toegankelijkheid en herbruikbaarheid van wetenschappelijke gegevens
te verbeteren. ``FAIR'' staat voor Findable (vindbaar), Accessible
(toegankelijk), Interoperable (interoperabel) en Reusable
(herbruikbaar). Deze principes zijn bedoeld om de uitdagingen aan te
pakken die ontstaan door de groeiende hoeveelheid digitale gegevens en
om de efficiëntie en effectiviteit van wetenschappelijk onderzoek te
vergroten.

De vier principes van FAIR-data zijn:

\begin{enumerate}
\def\labelenumi{\arabic{enumi}.}
\tightlist
\item
  Findable (vindbaar): Gegevens moeten gemakkelijk te vinden zijn, zowel
  voor mensen als voor computersystemen. Dit vereist het toekennen van
  unieke en persistente identificatoren aan datasets, het gebruik van
  gestandaardiseerde metadata en het opnemen van relevante informatie
  zoals de context, locatie en beschrijving van de gegevens.
\item
  Accessible (toegankelijk): Gegevens moeten openbaar beschikbaar zijn
  of op zijn minst toegankelijk zijn met duidelijke en goed
  gedefinieerde gebruiksvoorwaarden. Dit omvat het bieden van toegang
  tot de volledige dataset of ten minste tot de metadata en de
  mogelijkheid om de gegevens te downloaden of op te halen via
  geautomatiseerde interfaces.
\item
  Interoperable (interoperabel): Gegevens moeten worden gestructureerd
  op een manier die het mogelijk maakt om ze te integreren, te
  combineren en te analyseren met andere datasets. Dit houdt in dat
  gegevens moeten worden gepresenteerd met behulp van gemeenschappelijke
  standaarden, formaten en vocabulaires, en dat ze machineleesbaar
  moeten zijn om automatische verwerking mogelijk te maken.
\item
  Reusable (herbruikbaar): Gegevens moeten worden gedocumenteerd op een
  manier die begrip en hergebruik bevordert. Dit omvat het verstrekken
  van gedetailleerde en nauwkeurige metadata, het beschrijven van de
  gebruikte methoden en het documenteren van de datakwaliteit en de
  beperkingen. Bovendien moeten de gebruiksvoorwaarden en licenties
  duidelijk worden gecommuniceerd, zodat anderen de gegevens kunnen
  hergebruiken met passende erkenning en attributie.
\end{enumerate}

Het naleven van de FAIR-data principes bevordert transparantie,
samenwerking en innovatie in de wetenschap, en vergemakkelijkt de
replicatie en validatie van onderzoeksresultaten. Het stelt
wetenschappers, instellingen en datarepositoria in staat om waardevolle
gegevens effectief te beheren, te delen en te hergebruiken voor bredere
wetenschappelijke en maatschappelijke vooruitgang.

\hypertarget{wetenschappelijk-en-fair}{%
\section{Wetenschappelijk en FAIR?}\label{wetenschappelijk-en-fair}}

De relatie tussen de wetenschappelijke methode en FAIR-data ligt in het
streven naar betrouwbare en reproduceerbare wetenschappelijke
resultaten, evenals het bevorderen van openheid, transparantie en
samenwerking in het wetenschappelijk onderzoek.

De wetenschappelijke methode biedt een gestructureerde benadering om
hypotheses te testen en kennis te vergaren. Het omvat het ontwerpen en
uitvoeren van experimenten, het verzamelen en analyseren van gegevens,
en het trekken van conclusies op basis van objectieve bevindingen. De
wetenschappelijke methode is gebaseerd op het streven naar betrouwbare
en herhaalbare resultaten, waarbij transparantie en nauwkeurigheid van
cruciaal belang zijn.

Aan de andere kant richten de FAIR-data principes zich op het bevorderen
van de toegankelijkheid en herbruikbaarheid van wetenschappelijke
gegevens. Deze principes stellen richtlijnen vast voor het organiseren,
documenteren en delen van onderzoeksgegevens op een gestructureerde en
consistente manier. Door gegevens vindbaar, toegankelijk, interoperabel
en herbruikbaar te maken, kunnen andere onderzoekers deze gegevens
gebruiken om resultaten te reproduceren, verder onderzoek uit te voeren
of nieuwe inzichten te genereren.

Door de FAIR-data principes te volgen, kunnen wetenschappers hun
gegevens beter beheren en documenteren, waardoor ze effectiever kunnen
samenwerken en hun onderzoeksresultaten transparant kunnen delen. Dit
versterkt de betrouwbaarheid en validiteit van wetenschappelijk
onderzoek, en vergemakkelijkt de verificatie en replicatie van
bevindingen door andere wetenschappers.

Kortom, de wetenschappelijke methode en de FAIR-data principes delen het
gemeenschappelijke doel van het bevorderen van betrouwbaarheid,
transparantie en openheid in de wetenschap. Door de wetenschappelijke
methode te combineren met het gebruik van FAIR-data, kunnen
wetenschappers de kwaliteit en impact van hun onderzoek vergroten,
terwijl ze tegelijkertijd een robuuste basis leggen voor verdere
wetenschappelijke vooruitgang.

\hypertarget{wetenschappelijk-communiceren}{%
\section{Wetenschappelijk
communiceren}\label{wetenschappelijk-communiceren}}

Wetenschappelijk communiceren door middel van analytisch verhalen
vertellen, ook wel bekend als analytic storytelling, is een benadering
waarbij wetenschappelijke informatie op een narratieve en boeiende
manier wordt gepresenteerd. Hierbij wordt gebruik gemaakt van
verhaalstructuren en elementen om complexe gegevens en resultaten op een
toegankelijke manier over te brengen naar verschillende doelgroepen.

Analytische storytelling combineert de analytische en rigoureuze
aspecten van wetenschap met de kracht van verhalen. Het omvat de
volgende elementen:

\begin{enumerate}
\def\labelenumi{\arabic{enumi}.}
\tightlist
\item
  Identificeer het verhaal: Begin met het identificeren van het
  kernverhaal dat je wilt vertellen. Dit kan bijvoorbeeld het
  belangrijkste wetenschappelijke resultaat zijn dat je wilt
  communiceren of de ontdekking die je hebt gedaan.
\item
  Kies een narratieve structuur: Selecteer een geschikte narratieve
  structuur, zoals een reisverhaal, een heldenreis of een
  probleem-oplossingsverhaal. Het doel is om een logische en meeslepende
  verhaallijn te creëren die de lezer of luisteraar aanspreekt.
\item
  Gebruik personages: Introduceer personages, zoals wetenschappers,
  proefpersonen of historische figuren, om het verhaal menselijker en
  persoonlijker te maken. Dit helpt de lezers zich te identificeren met
  de wetenschappelijke context en betrokken te raken bij het verhaal.
\item
  Gebruik visuele elementen: Ondersteun het verhaal met visuele
  elementen, zoals afbeeldingen, grafieken, diagrammen of video's. Deze
  visuele hulpmiddelen kunnen helpen om complexe gegevens begrijpelijker
  te maken en de boodschap te versterken.
\item
  Maak gebruik van emotie en spanning: Creëer emotie en spanning in het
  verhaal om de interesse van het publiek te wekken en vast te houden.
  Dit kan gedaan worden door uitdagingen, onverwachte wendingen of
  persoonlijke anekdotes toe te voegen.
\item
  Wees duidelijk en beknopt: Zorg ervoor dat de wetenschappelijke
  informatie duidelijk en beknopt wordt gecommuniceerd. Vermijd jargon
  en technisch taalgebruik en leg complexe concepten op een
  begrijpelijke manier uit.
\item
  Verbind met de bredere betekenis: Breng de wetenschappelijke
  bevindingen in verband met bredere betekenis en relevantie,
  bijvoorbeeld in termen van maatschappelijke impact, toekomstig
  onderzoek of praktische toepassingen.
\end{enumerate}

Door analytische storytelling toe te passen, kunnen wetenschappers hun
onderzoek op een boeiende en begrijpelijke manier presenteren aan
verschillende doelgroepen, waaronder niet-wetenschappers. Dit draagt bij
aan een betere verspreiding van wetenschappelijke kennis, vergroot het
begrip en de waardering voor wetenschap, en bevordert de dialoog tussen
wetenschappers en de samenleving.

\bookmarksetup{startatroot}

\hypertarget{wetenschappelijke-methode-1}{%
\chapter{Wetenschappelijke methode}\label{wetenschappelijke-methode-1}}

De wetenschappelijke methode is een systematische benadering die wordt
gebruikt in de wetenschap om kennis te vergaren en hypothesen te testen.
Het omvat een reeks stappen die wetenschappers volgen om betrouwbare en
objectieve resultaten te verkrijgen. Over het algemeen bestaat de
wetenschappelijke methode uit de volgende stappen:

\begin{enumerate}
\def\labelenumi{\arabic{enumi}.}
\tightlist
\item
  Observatie: Waarnemen van een fenomeen of verschijnsel en het stellen
  van een vraag over hoe of waarom dit gebeurt.
\item
  Hypothese: Formuleren van een verklaring of een voorlopige antwoord op
  de vraag die gebaseerd is op bestaande kennis en logisch redeneren.
\item
  Voorspelling: Afleiden van specifieke voorspellingen uit de hypothese
  die getest kunnen worden.
\item
  Experiment: Ontwerpen en uitvoeren van gecontroleerde experimenten om
  de voorspellingen te testen. Het experiment moet zorgvuldig worden
  opgezet, met controlevariabelen en experimentele groepen, om
  nauwkeurige en reproduceerbare resultaten te verkrijgen.
\item
  Data-analyse: Verzamelen en analyseren van de gegevens die tijdens het
  experiment zijn verzameld, vaak met behulp van statistische methoden.
\item
  Conclusie: Op basis van de resultaten van het experiment, beoordelen
  of de hypothesen worden ondersteund of weerlegd. Het kan nodig zijn om
  het experiment te herhalen of verder onderzoek uit te voeren om tot
  een definitieve conclusie te komen.
\item
  Communicatie: Het rapporteren van de resultaten, inclusief de methode,
  bevindingen en conclusies, aan de wetenschappelijke gemeenschap via
  publicaties, presentaties en discussies.
\end{enumerate}

De wetenschappelijke methode is een iteratief proces, waarbij nieuwe
observaties en experimenten kunnen leiden tot herziening van hypothesen
en nieuwe vragen, waardoor de wetenschappelijke kennis voortdurend
groeit en evolueert.

\bookmarksetup{startatroot}

\hypertarget{zoekmachines-voor-literuuronderzoek}{%
\chapter{Zoekmachines voor
literuuronderzoek}\label{zoekmachines-voor-literuuronderzoek}}

Hoe zoeken?

https://paperpile.com/g/academic-search-engines/

\begin{itemize}
\item
  Google scholar: \url{https://scholar.google.com/}
\item
  BASE: \url{https://www.base-search.net/}
\item
  CORE: \url{https://core.ac.uk/}
\item
  Science.gov: \url{http://science.gov/}
\item
  Semantic Scholar: \url{https://www.semanticscholar.org/}
\item
  RefSeek: \url{https://www.refseek.com/}
\end{itemize}

\bookmarksetup{startatroot}

\hypertarget{mendeley-voor-referentiebeheer}{%
\chapter{Mendeley voor
referentiebeheer}\label{mendeley-voor-referentiebeheer}}

\ldots{}

Online referentie manager: \url{https://www.mendeley.com}

Eventueel kan ook \url{https://www.scribbr.com}

\bookmarksetup{startatroot}

\hypertarget{python-voor-data-analyse}{%
\chapter{Python voor data-analyse}\label{python-voor-data-analyse}}

Informaticawetenschappen.

Python tutorial gebaseerd op
\url{https://wesmckinney.com/book/preface.html}

\bookmarksetup{startatroot}

\hypertarget{latex-voor-publicaties}{%
\chapter{LaTeX voor publicaties}\label{latex-voor-publicaties}}

Online editor: \url{https://www.overleaf.com}

Hoe een goede wetenschappelijke tekst schrijven?

\begin{itemize}
\item
  Methodes: \url{https://plos.org/resource/how-to-write-your-methods/}
\item
  Conclusie: \url{https://plos.org/resource/how-to-write-conclusions/}
\item
  Abstract:
  \url{https://plos.org/resource/how-to-write-a-great-abstract/}
\item
  Titel: \url{https://plos.org/resource/how-to-write-a-great-title/}
\item
  Klaarmaken ter publicatie:
  \url{https://plos.org/resource/how-to-edit-your-work/}
\end{itemize}

\bookmarksetup{startatroot}

\hypertarget{slides-voor-presentaties}{%
\chapter{Slides voor presentaties}\label{slides-voor-presentaties}}

reveal.js is een open source HTML-presentatieraamwerk. Het is een tool
waarmee iedereen met een webbrowser gratis volledig uitgeruste en
prachtige presentaties kan maken.

Online editor: \url{https://slides.com/}

\bookmarksetup{startatroot}

\hypertarget{project-1-logaritmische-schaal}{%
\chapter*{Project 1: logaritmische
schaal}\label{project-1-logaritmische-schaal}}
\addcontentsline{toc}{chapter}{Project 1: logaritmische schaal}

\markboth{Project 1: logaritmische schaal}{Project 1: logaritmische
schaal}

\bookmarksetup{startatroot}

\hypertarget{project-2-waterbalans}{%
\chapter*{Project 2: waterbalans}\label{project-2-waterbalans}}
\addcontentsline{toc}{chapter}{Project 2: waterbalans}

\markboth{Project 2: waterbalans}{Project 2: waterbalans}

\bookmarksetup{startatroot}

\hypertarget{bibliografie}{%
\chapter*{Bibliografie}\label{bibliografie}}
\addcontentsline{toc}{chapter}{Bibliografie}

\markboth{Bibliografie}{Bibliografie}

\hypertarget{refs}{}
\begin{CSLReferences}{0}{0}
\end{CSLReferences}



\end{document}
